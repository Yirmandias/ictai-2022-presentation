\section{Deliberative agent (7 min)}

\begin{frame}{What is a robotic agent}

    \begin{columns}[T]
        \begin{column}{0.3\textwidth}
            An entity

            ~

            \includegraphics[width = 0.7\textwidth]{images/icons8-robot-gustav-500.png}
        \end{column}
        \begin{column}{0.7\textwidth}
            \center Capable of :
            \pause
            \begin{enumerate}
                \item Perceiving its environment
                \pause
                \item Modifying its environment (actions)
                \pause
                \item \textbf{Deliberating:} Reasoning about its \textbf{skills} in order to fulfill a goal
            \end{enumerate}
        \end{column}
    \end{columns}

 
\end{frame}

\begin{frame}{Represent the skills of an agent}

\begin{columns}[T]
    \begin{column}{0.55\textwidth}

        Robot behavior = \textbf{operational model}:
        \pause
        \small
        \begin{itemize}
            \item Executable program
            \item General purpose language
        \end{itemize}
        
        ~
        \pause
        \normalsize
        Acting domain \textbf{$A_\Delta (A, T, M_t)$}: Hierarchical operational models
        \small
        \pause
        \begin{itemize}
        
        
         \item[$A$] \textit{(low-level skills)}: primitive tasks
         \pause
         
         \textit{\footnotesize(move, grasp, look)}
         \pause
         \item[$T$] \textit{(high-level skills)}: abstract tasks
         \pause

         \textit{\footnotesize(set the table, prepare a coffee)}
         \pause
         \item[$M_t$] \textit{("Know how")}: methods
         pre-conditions, body (operational model)

         
     \end{itemize}
    \end{column}
    \begin{column}{0.45\textwidth}
        \begin{figure}
            \begin{tikzpicture}
                \node[draw,ellipse, ultra thick] (t) {\textit{open door}} [sibling distance = 3.5cm]
                  child {node[draw, ultra thick] (m1) {$m_1$} edge from parent [dashed]
                  child {node[draw,rounded corners, ultra thick, solid] (a1) {$push$} edge from parent
                  }} 
                  child {node[draw, ultra thick] (m2) {$m_2$} edge from parent [dashed] [sibling distance = 1.5cm]
                  child {node[draw, rounded corners, solid, ultra thick] (a2) {grab handle} edge from parent [solid]}
                  child {node[draw, rounded corners, solid, ultra thick] {$pull$} edge from parent [solid]}};
                \node[right = 0em of t] {$\in T$};
                \node[right = 0em of m1] {$\in M_t$};
                \node[right = 0em of m2] {$\in M_t$};
                \node[right = 0em of a1] {$\in A$};

            \end{tikzpicture}
            \caption{Example of hierarchy for the \textit{task} \textit{open door}}

            
        \end{figure}
    \end{column}
\end{columns}
    
\end{frame}
\begin{frame}{Refinement Acting Engine (RAE)\cite{ghallabAutomatedPlanningActing2016} : Deliberation algorithms using hierarchical operational models}
\begin{columns}[T]
    \begin{column}{0.35\textwidth}
    \pause
    \setlength{\leftmargini}{-1pt}
    %\setlength{\parsep}{1pt}
    %\setlength{\parskip}{0pt}
    RAE features:
    \small
    \begin{itemize}
        \item Perform multiple tasks in parallel
        \pause
        \item Automated deliberation
        \begin{itemize}
            \setlength{\leftmargini}{-1pt}
            \pause
            \item Refinement of task into a method
            \pause
            \item Instantiating of \textbf{arbitrary} variables
        \end{itemize}
    \end{itemize}
    \end{column}
    \begin{column}{0.65\textwidth}
        
        \pause
        Algorithms:
        \small
        \pause
        \begin{itemize}
            \setlength{\leftmargini}{-1pt}
            \item \textbf{Main:} 
            \begin{itemize}
                \item Receive $\tau$ (task or event);
                
                add it to the \textbf{agenda} (ongoing tasks)
                \pause
                \item Refine $\tau$: \textbf{Select} an applicable method $m$ for $\tau$
                \pause
                \item \textbf{Progress} $m$
            \end{itemize}
            \pause
            \item \textbf{Progress:}
                \begin{itemize}
                \pause
                    \item Monitor execution of $m$.
                \pause
                    \item Refine subtasks in $m$.    
                \pause
                    \item Monitor execution of subtasks.
                \pause
                    \item \textbf{Retry} $\tau$ in case of \emph{failure}:
                
                Call \textbf{Select} to get a new method;
                
                \textbf{Progress} the new method.
                \end{itemize}
                \pause
        \end{itemize}
    \end{column}
\end{columns}

    
\end{frame}

\begin{frame}{Improve the refinement using planning}
    \begin{center}
        
    Role of Select:

    ~

    \begin{tikzpicture}[thick,scale=0.8, every node/.style={scale=0.8}]
    \node[draw = black, very thick,
        minimum width = 6em,
        minimum height = 3em,
        rounded corners,
        text = black,
        align=center,] (F) at (0,0) {\textbf{Select}};
        
    \node[left= 2em of F] (i) {\textbf{$\tau(p_1,\dots,p_n), \xi (state)$}};
    \node[right= 2em of F] (o) {\textbf{$m(p_1, \dots, p_n,\dots,p_m)$}};
    \path[->]
    (i) edge (F)
    (F) edge (o);

    \end{tikzpicture}
    

    %$Select(\tau, p_1,\dots,p_n) \rightarrow \{m_s, p_1, \dots, p_n,\dots,p_m\}$
    \end{center}

    \pause
    Techniques:
    \begin{itemize}

    \item Greedy (Basic RAE functioning): arbitrary applicable method
    \pause
    
    \underline{Problem:} Does not take into account future refinements (can lead into dead-locks)
    \pause
    \item \textbf{look-ahead(planning)} : capacity to project the system from the current state to possible future state

    \end{itemize}
\end{frame}
\begin{frame}{How to use planning in RAE}
    %Make a high-level choice based on future choices the agent will have to make and its own capabilities to modify its environment.
    Requires:
    \begin{itemize}
        \item A planner
        \item Descriptive model of the agent skills : describes the set of states that may result from performing tasks.
    \end{itemize} 
\pause
    Descriptive models shortcomings:
    \begin{itemize}
        \item Made limited dedicated languages: PDDL \cite{foxPDDL2ExtensionPDDL2003}, HDDL \cite{hollerHDDLExtensionPDDL2020}, ANML \cite{smith2008anml},\dots
        \item operational model $\not\equiv$ descriptive model $\rightarrow$ Harder design
    \end{itemize}

    ~
    \pause
    \centering
    \begin{tikzpicture}
        \node[align = center] (O) {Operational\\ model \\};
        \node[right=6em of O, align = center] (D) {Descriptive\\ Model \\(chronicles)};
        \draw[-latex, ultra thick] (O.east)-- node[above, midway] (c) {\textbf{Extracting}} (D.west);
    \end{tikzpicture}

    \pause
    Plan with \textbf{Aries} (hierarchical LCP\cite{bit-monnotConstraintBasedEncodingDomainIndependent2018} extension):

\end{frame}

\begin{frame}[c]{An analyzable acting language to extract descriptive models}
    \begin{columns}[c]
        \begin{column}{0.45\textwidth}
            Acting language Requirements:
            \begin{itemize}
                \item General purpose language
                \item Identified and simple semantic
            \end{itemize}
        \end{column}
        \begin{column}[c]{0.1\textwidth}
            \centering
            $\rightarrow$
        \end{column}
        \begin{column}{0.5\textwidth}
            \pause
            Lisp dialect (Scheme variant) \cite{moretti1979lambda}

            Perks:
            \begin{itemize}
                \item Few primitives
                \item Immutable
                \item Functional
                \item Pure
            \end{itemize}
        \end{column}
    \end{columns}
\end{frame}

\begin{frame}{Synthesis of the proposition}
        \centering
        Task refinement guided by planning:

        ~
        
        \begin{tikzpicture}
            \node[draw,
            rounded corners,
            minimum width = 5em,
            minimum height = 3em] (select) {\LARGE Select};
            \node[draw,
            rounded corners,
            right = 6em of select,
            minimum width = 5em,
            minimum height = 3em] (aries) {\LARGE Aries};
            \path[->, every node/.style={font=\sffamily\small}]
            (select) edge[bend right] node [right, below] {$P_\Delta, t(p_1,\dots,p_n)$} (aries)
            (aries) edge[bend right] node [right, above] {$Plan(m_s, m_2,\dots,m_n)$} (select)
            ;
        \end{tikzpicture}

        ~
    \pause
        Planning domain extraction from operational models:

        ~

        \begin{tikzpicture}
            \node (acting) {\LARGE $A_\Delta$};
            
            \node[draw,
            rounded corners,
            right= 3em of acting,
            minimum height = 3em,
            minimum width = 6em] (ext) {\Large Extraction};
        
            \node[right = 3em of ext] (planning) {\LARGE $P_\Delta$ };
            
            \path[->, every node/.style={font=\sffamily\small}]
            (acting) edge (ext)
            (ext) edge (planning);
          \end{tikzpicture}

\end{frame}


