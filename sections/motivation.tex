\section{Motivation}
\begin{frame}{What is a robotic agent}

    \begin{columns}[T]
        \begin{column}{0.3\textwidth}
            An entity

            ~

            \includegraphics[width = 0.7\textwidth]{images/icons8-robot-gustav-500.png}
        \end{column}
        \begin{column}{0.7\textwidth}
            \center Capable of :
            \pause
            \begin{enumerate}
                \item Perceiving its environment
                \pause
                \item Modifying its environment (actions)
                \pause
                \item \textbf{Deliberating:} Reasoning about its \textbf{skills} in order to fulfill a goal
            \end{enumerate}
        \end{column}
    \end{columns}

 
\end{frame}

\begin{frame}{Skills of an agent}

\begin{columns}[T]
    \begin{column}{0.55\textwidth}

        \begin{itemize}
            \item Elementary capabilities : move, grasp, look, \dots
            \pause
            \item Skills : executable programs (operational models): set the table, \dots
            \pause
            \item Robot behavior = composition of skills
        \end{itemize}
        
        ~
        \pause
        Acting domain \textbf{$A_\Delta (A, T, M_t)$}: Hierarchical operational models
        \small
        \pause
        \begin{itemize}
        
        
         \item[$A$] : primitive tasks
         \pause
         \item[$T$] : abstract tasks
         \pause
         \item[$M_t$] : methods: pre-conditions, body (operational model)
         
     \end{itemize}
    \end{column}
    \pause
    \begin{column}{0.45\textwidth}
        \begin{figure}
            \begin{tikzpicture}
                \node[draw,ellipse, ultra thick] (t) {\textit{open door}} [sibling distance = 3.5cm]
                  child {node[draw, ultra thick] (m1) {$m_1$} edge from parent [dashed]
                  child {node[draw,rounded corners, ultra thick, solid] (a1) {$push$} edge from parent
                  }} 
                  child {node[draw, ultra thick] (m2) {$m_2$} edge from parent [dashed] [sibling distance = 1.5cm]
                  child {node[draw, rounded corners, solid, ultra thick] (a2) {grab handle} edge from parent [solid]}
                  child {node[draw, rounded corners, solid, ultra thick] {$pull$} edge from parent [solid]}};
                \node[right = 0em of t] {$\in T$};
                \node[right = 0em of m1] {$\in M_t$};
                \node[right = 0em of m2] {$\in M_t$};
                \node[right = 0em of a1] {$\in A$};

            \end{tikzpicture}
            \caption{Example of hierarchy for the \textit{task} \textit{open door}}

            
        \end{figure}
    \end{column}
\end{columns}
    
\end{frame}

\begin{frame}{Refinement Acting Engine (RAE)\footnote{Automated Planning and Acting \cite{ghallabAutomatedPlanningActing2016}}: deliberation algorithms using hierarchical operational models}
    RAE features:
    \begin{itemize}
        \item Perform multiple tasks in parallel
        \pause
        \item Automated deliberation: Refinement of task into method and choice of parameters
    \end{itemize}
    \centering
    \begin{tikzpicture}
        \node[] (task) {$\tau(p_1,\dots,p_n)$};
        \node[below = 2em of task] (method) {$m(p_1,\dots,p_n, \dots, p_m)$};
        \path[->] (task) edge node[right, midway] {refinement} (method);
    \end{tikzpicture}
        
\end{frame}
    

\begin{frame}{Refinement Acting Engine (RAE) : algorithms}
    \begin{columns}[T]
  
        \begin{column}{0.65\textwidth}
            
            Algorithms:
            \small
            \begin{itemize}
                \setlength{\leftmargini}{-1pt}
                \onslide<2->
                \item \textbf{Main:} 
                \begin{itemize}
                    \item Receive $\tau$ (task or event);
                    
                    add it to the \textbf{agenda} (ongoing tasks)
                    \onslide<3->
                    \item Refine $\tau$: \textbf{Select} an applicable method $m$ for $\tau$
                    \onslide<4->
                    \item \textbf{Progress} $m$
                \end{itemize}
                \onslide<5->
                \item \textbf{Progress:}
                    \begin{itemize}
                        \onslide<6->
                        \item Monitor execution of $m$.
                        \onslide<7->
                        \item Refine subtasks in $m$.    
                        \onslide<8->
                        \item Monitor execution of subtasks.
                        \onslide<9->
                        \item \textbf{Retry} $\tau$ in case of \emph{failure}:
                    
                    Call \textbf{Select} to get a new method;
                    
                    \textbf{Progress} the new method.
                    \end{itemize}
            \end{itemize}
        \end{column}
        \begin{column}{0.35\textwidth}
            \begin{tikzpicture}
                \onslide<2>
                \node[draw,ellipse, ultra thick] (t) {$\tau$};
                \onslide<3-9>
                \node[draw,ellipse, ultra thick, color = orange] (t) {$\tau$};
                \onslide<10>
                \node[draw,ellipse, ultra thick, color = green] (t) {$\tau$};
                \onslide<3>
                \node[draw, ultra thick, below= 2em of t, xshift = -3em] (m1) {$m_1$};
                \node[draw, ultra thick, below= 2em of t, xshift = 3em] (m2) {$m_2$};
                \onslide<4->
                \node[draw, ultra thick, below= 2em of t, xshift = -3em, color = orange] (m1) {$m_1$};
                \node[draw, ultra thick, below= 2em of t, xshift = 3em, color = gray] (m2) {$m_2$};
                \onslide<10>
                \node[draw, ultra thick, below= 2em of t, xshift = -3em, color = green] (m1) {$m_1$};
                \onslide<3->
                \path[-] (t) edge (m1) (t) edge (m2);
                \onslide<5>
                \node[draw,rounded corners, ultra thick, solid, below = 2em of m1, xshift = -1.5em] (a1) {$a1$};
                \onslide<6>
                \node[draw,rounded corners, ultra thick, solid, below = 2em of m1, xshift = -1.5em, color = orange] (a1) {$a1$};
                \onslide<7->
                \node[draw,rounded corners, ultra thick, solid, below = 2em of m1, xshift = -1.5em, color = green] (a1) {$a1$};
                \onslide<5-7>
                \node[draw,ellipse, ultra thick, solid, below = 2em of m1, xshift = 1.5em] (t2) {$\tau_s$};
                \onslide<8->
                \node[draw,ellipse, ultra thick, solid, below = 2em of m1, xshift = 1.5em, color = orange] (t2) {$\tau_s$};
                \onslide<10>
                \node[draw,ellipse, ultra thick, solid, below = 2em of m1, xshift = 1.5em, color = green] (t2) {$\tau_s$};
  
  
                \onslide<5->
                \path[-] (m1) edge (a1) (m1) edge(t2);
  
  
                \onslide<7>
                \node[draw, ultra thick, below = 2em of t2, xshift = -1.5em] (m3) {$m_3$};
                \node[draw, ultra thick, below = 2em of t2, xshift = 1.5em] (m4) {$m_4$};
                \path[-] (t2) edge  (m3) (t2) edge (m4);
                
                \onslide<8>
                \node[draw, ultra thick, below = 2em of t2, xshift = -1.5em, color = orange] (m3) {$m_3$};
                \node[draw, ultra thick, below = 2em of t2, xshift = 1.5em, color = gray] (m4) {$m_4$};
                \path[-] (t2) edge  (m3) (t2) edge (m4);
  
                \onslide<9->
                \node[draw, ultra thick, below = 2em of t2, xshift = -1.5em, color = red] (m3) {$m_3$};
                \node[draw, ultra thick, below = 2em of t2, xshift = 1.5em, color = orange] (m4) {$m_4$};
                \path[-] (t2) edge  (m3) (t2) edge (m4);
                \onslide<10>
                \node[draw, ultra thick, below = 2em of t2, xshift = 1.5em, color = green] (m4) {$m_4$};
                
                \onslide<8->
                \node[below = 2em of m3, xshift = -1em] (e3) {};
                \node[below = 2em of m3, xshift = 1em] (e4) {};
                \path[-] (m3) edge  (e3) (m3) edge (e4);
  
                \onslide<8->
                \node[below = 2em of m4, xshift = -1em] (e5) {};
                \node[below = 2em of m4, xshift = 1em] (e6) {};
                \path[-] (m4) edge  (e5) (m4) edge (e6);
      
            \end{tikzpicture}
        \end{column}
    \end{columns}    
\end{frame}

\begin{frame}{Extending its definition for concurrency}
    \begin{columns}
        \begin{column}{0.5\textwidth}
            What lacks in the definition of RAE
        \end{column}
        \begin{column}{0.5\textwidth}
            What we want to add
        \end{column}
    \end{columns}
\end{frame}