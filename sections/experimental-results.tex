\section{Results}
\subsection{Multi-agent model for the factory simulator}

\begin{frame}[fragile]{Multi-agent model}
%     \begin{columns}
%         \begin{column}{0.5\textwidth}
%             \centering
%             \includegraphics[width = 0.3\textwidth]{images/godot/package.png}
            
%             \Large $\times n$
%         \end{column}
%         \begin{column}{0.5\textwidth}
%             \centering
%             \includegraphics[width = 0.3\textwidth]{images/godot/robot_texture.png}
            
%             \LARGE $\times m$
%         \end{column}
%     \end{columns}
% \pause
\begin{columns}[t]
    \begin{column}{0.5\textwidth}
        A unique high-level task to manage all packages and robots:

    \scriptsize
    \setlength{\leftmargini}{0pt}
    \lstset{columns=fullflexible}
\begin{lstlisting}[language = lisp]
(do
;declare robots and machines as resources
 (mapf new-resource (instances robot))
 (mapf new-resource (instances machine))
;process all packages in parallel
 (define handle-packages
  (mapf (lambda (?p)
    (async (t_process_package ?p)))
   (instances packages)))
;monitor in parallel batteries of robots
 (define handle-robots
  (mapf (lambda (?r)
    (async (t_check_battery ?r)))
   (instances robots)))
;await that all packages are processed
 (mapf await handle-packages))
        \end{lstlisting}           
    \end{column}

\pause
    \begin{column}{0.5\textwidth}
        Process package task:
        \tiny
        \setlength{\leftmargini}{0pt}
        \lstset{columns=fullflexible}
        \begin{lstlisting}
(def-task t_process_package (:params (?p package)))

(def-method m_process_to_do_r
 (:task t_process_package)
 (:params (?p package))
 (:pre-conditions (! (null?(package.processes_list ?p))))
 (:body
  (do
   (define ?m
    (arbitrary (find_machines_for_process
     (caar (unzip (package.processes_list ?p))))))
   (define ?r (arbitrary (instances robot) rand-element))
   (define h_r (acquire ?r))
   (define h_m (acquire ?m))
   (t_carry_to_machine ?r ?p ?m)
   (release h_r)
   (process ?m ?p)
   (t_process_package ?p))))                  

(def-method m_no_more_process
 (:task t_process_package)
 (:params (?p package))
 (:pre-conditions (null? (package.processes_list ?p)))
 (:body
  (do
   (define ?r (arbitrary (instances robot) rand-element))
   (define h (acquire ?r))
   (t_carry_to_machine ?r ?p
    (find_output_machine)))))            
        \end{lstlisting}
    \end{column}
\end{columns}
 
\end{frame}

\begin{frame}{Running example for several packages}
    \centering

    \includegraphics[width = 0.8\textwidth]{images/gobot-multi-robot.png}
    \Large

    Video example: \href{https://youtu.be/8z2aUsWTq4k}{https://youtu.be/8z2aUsWTq4k}
\end{frame}

\subsection{Validation on job shop configurations}
\begin{frame}[fragile]{Validation on job shop problems}
%\centering
%Outline the features of the new system and the language
%Organiser le passage de plusieurs tâches sur des machines.
%transport des paquets entre les machines
Job shop : $n$ packages should be processed on $m$ machines

~~

\pause
Restriction of the simulator:
\begin{itemize}
    \item All packages are available at the beginning of the simulation
    \item Each machine can only do one process
\end{itemize}

\pause
Robotic system mission:
\begin{itemize}
    \pause
    \item Schedule packages passages on machines to optimize the total processing time.
    \item Take into account robot displacement times and battery recharge.
\end{itemize}
\end{frame}


\begin{frame}[fragile]{Comparison of different reactive resource acquisition strategies}
    Several resource allocation strategies in task \textit{t\_process\_package ?p}:

~~

\pause
    \begin{columns}
    \begin{column}{0.33\textwidth}
        \small
        Random (R):
        \tiny
        \setlength{\leftmargini}{0pt}
        \lstset{columns=fullflexible}
        \begin{lstlisting}
(do ...
  (define ?r 
    (arbitrary (instances robot)
    rand-element))
  (define h_r (acquire ?r))
  (define h_m (acquire ?m))
    ...)
        \end{lstlisting}
    \end{column}

    \pause

    \begin{column}{0.33\textwidth}
        \small
        First Available (FA):
        \tiny
        \setlength{\leftmargini}{0pt}
        \lstset{columns=fullflexible}
        \begin{lstlisting}
(do ...
  (define h_m (acquire ?m))
  (define h_r (acquire-in-list
      (instances robot)))
  (define ?r (first h_r))
  ...)
        \end{lstlisting}   
    \end{column}

    \pause

    \begin{column}{0.33\textwidth}
        \small
        FA+\textit{Longest Remaining Processing Time First} (FA+LRPTF)
        \tiny
        \setlength{\leftmargini}{0pt}
        \lstset{columns=fullflexible}
        \begin{lstlisting}
(do ...
  (define r-time (remaining-time ?p))
  (define h_m (acquire ?m 
      `(:priority ,r-time)))
  (define h_r  
    (acquire-in-list 
      (instance robot)
      `(:priority ,r-time)))
  (define ?r (first h_r))
  ...)
        \end{lstlisting}   
    \end{column}
\end{columns}
\end{frame}




\newcommand{\calcrowmean}{
    \def \rowmean{0}
    \pgfmathparse{\pgfkeysvalueof{/pgfplots/table/summary statistics/end index}-\pgfkeysvalueof{/pgfplots/table/summary statistics/start index}+1}
    \edef\numberofcols{\pgfmathresult}
            % ... loop over all columns, summing up the elements
    \pgfplotsforeachungrouped \col in {1,2,3,4,5,6,7,8,9,10}% in {\pgfkeysvalueof{/pgfplots/table/summary statistics/start index},...,\pgfkeysvalueof{/pgfplots/table/summary statistics/end index}}
    {
        
        \typeout{col = \col}
        
        \pgfmathparse{\rowmean+\thisrowno{\col}/\numberofcols}
        \edef \rowmean{\pgfmathresult}
    }
}
\newcommand{\calcstddev}{
    \def\rowstddev{0}
    \calcrowmean
    \pgfplotsforeachungrouped \col in {1,2,3,4,5,6,7,8,9,10}
    % {\pgfkeysvalueof{/pgfplots/table/summary statistics/start index},...,\pgfkeysvalueof{/pgfplots/table/summary statistics/end index}}
    {
        \pgfmathparse{\rowstddev+(\thisrowno{\col}-\rowmean)^2/(\numberofcols-1)}
        \edef\rowstddev{\pgfmathresult}
    }
    \pgfmathparse{sqrt(\rowstddev)}
}
\newcommand{\calcstderror}{
    \calcrowmean
    \calcstddev
    \pgfmathparse{sqrt(\rowstddev)/sqrt(\numberofcols)}
}

\pgfplotstableset{
    summary statistics/start index/.initial=1,
    summary statistics/end index/.initial=10,
    create col/mean/.style={
        /pgfplots/table/create col/assign/.code={% In each row ... 
            \calcrowmean
            \pgfkeyslet{/pgfplots/table/create col/next content}\rowmean
        }
    },
    create col/standard deviation/.style={
        /pgfplots/table/create col/assign/.code={% In each row ... 
            \calcstddev
            \pgfkeyslet{/pgfplots/table/create col/next content}\pgfmathresult
        }
    },
    create col/standard error/.style={
        create col/assign/.code={% In each row ... 
            \calcstderror
            \pgfkeyslet{/pgfplots/table/create col/next content}\pgfmathresult
        }
    }
}

\pgfplotstableset{
    create on use/mean/.style={create col/mean},
    create on use/stddev/.style={create col/standard deviation},
    create on use/stderror/.style={create col/standard error}
}



\begin{frame}[fragile]{Comparison of reactive strategies on 6x6 job shop scenarios}

    \begin{tikzpicture}
        \begin{axis}[
                    /pgf/number format/.cd,
                        use comma,
                    height = 6cm,
                    width = \linewidth,
                    ymajorgrids,
                    ylabel={Time in seconds},
                    xlabel={Problems},
                    ymin = 0,
                    ymax= 220,
                    ybar=0pt,
                    bar width=12pt,
                    enlarge x limits = 0.3,
                    nodes near coords,
                    point meta=explicit symbolic,
                    scatter/position=absolute,
                    every node near coord/.style={
                            at={(\pgfkeysvalueof{/data point/x},1.8)},
                            anchor=south,
                        },
                    bar shift=0pt,
                    xtick={0,1,2,3},
                    xticklabels={p1,p2,p3,p4},
                    x tick label style={rotate=45,anchor=east},
                    legend cell align = {left},
                    legend pos = north west,
                    legend image post style={scale=0.4},
                    legend style={font = \footnotesize},
                ]
        \addplot+[bar shift = -12pt]
        plot[
                %smooth,
                error bars/.cd,
            y dir=both,
            y explicit
        ]
        table[
                x=Problem,
                y=mean,
                y error=stderror
        ]
        {datas/jobshop_greedy.dat};
        \addplot+
        [bar shift = +0pt]
        plot[
                %smooth,
                error bars/.cd,
            y dir=both,
            y explicit
        ]
        table[
                x=Problem,
                y=mean,
                y error=stderror
        ]
        {datas/jobshop_advanced.dat};
        \addplot+[bar shift = +12pt]
        plot[
                %smooth,
                error bars/.cd,
            y dir=both,
            y explicit
        ]
        table[
                x=Problem,
                y=mean,
                y error=stderror
        ]
        {datas/jobshop_advanced_lrptf.dat};
    
        \legend{
                    R,
                    FA,
                    FA+LRPTF}
    \end{axis}
    \end{tikzpicture}
    
    \pause

    Future work: improve the system with \textbf{anticipation} capabilities \textbf{(planning)}
  
\end{frame}