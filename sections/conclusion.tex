\section{Conclusion \& perspectives (2 min)}
\begin{frame}{On the paper}
    
\end{frame}
\begin{frame}{Perspectives}
    
\end{frame}

\begin{frame}{Conclusion}
    \begin{itemize}
        \item Acting language with:
        \begin{itemize}
            \item Concurrency handling
            \item Asynchronous execution
            \item Conversion to descriptive model
        \end{itemize}
        \pause
        \item Acting system:
        \begin{itemize}
            \item Using the custom language
            \item Capable of using planning to guide the refinement of task
            \item Reactively handle jobshop
        \end{itemize}
    \end{itemize}
\end{frame}
    
\begin{frame}{A little perspective on the results}
\begin{itemize}
    \item Most domains are simple and mostly problems of graph exploration.
    \pause
    \item Results show that planning is better than greedy, but comparison with, for example UPOM,  on the same domains could be interesting.
    \pause
    \item No error in the execution so no new plan is required.
\end{itemize}
\end{frame}

\begin{frame}{Short-term challenges}
    \begin{itemize}
        \item Handle all core operators, in particular async and await
        \pause
        \item Formalization of resource management in the acting engine
        \pause
        \item Graph Flow instead of SSA => branching representation
        \pause
        \item Errors modelled in descriptive models
    \end{itemize}
\end{frame}

\begin{frame}{Mid-term challenges}
    \begin{itemize}
        \item Resolve Jobshop and Flowshop problems in dynamic environment
        \pause
        \item Use planning both planning as anticipation, and in continuous planning
    \end{itemize}
\end{frame}