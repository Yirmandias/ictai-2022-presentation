\section{Deliberative agent (7 min)}

\begin{frame}{Agent definition}
    \center An entity capable of :
    \pause
    \begin{enumerate}
        \item Perceiving its environment
        \pause
        \item Modifying its environment (actions)
        \pause
        \item \textbf{Reasoning about its knowledge}
    \end{enumerate}

    \centering
    \begin{tikzpicture}[thick,scale=0.8, every node/.style={scale=0.8}]
    \node[draw = black, very thick,
        minimum width = 12em,
        minimum height = 5em,
        rounded corners,
        text = black,
        align=center,] (F) at (0,0) {\textbf{Reasoning}\\ Knowledge};
        
    \pause
    \node[above=1em of F, xshift = -12em] (t1) {\Large\textbf{Inputs}};
    \node[above=1em of F, xshift = 12em] (t2) {\Large \textbf{Outputs}};
    \pause
    \node[yshift = 1.5em, xshift = -12em] (t3) {\textbf{Perception datas}};
    \node[yshift = -1.5em, xshift = -12em] (t4) {\textbf{Objectives}};
    \draw[-latex, ultra thick] (-6,0)--(F.west);
    \pause
    \node[xshift = 15em] (t5) {\textbf{Actions}};
    \draw[-latex, ultra thick] (F.east) -- (t5);
    \end{tikzpicture}  
\end{frame}

\begin{frame}{Skills domain}
    \begin{center}
        \large
        Acting domain \textbf{$A_\Delta (A, T, M_t)$}:
        
        hierarchical operational models
    \end{center}
\begin{columns}
    \begin{column}{0.65\textwidth}
        
     \begin{itemize}
         \item[$A$] \textit{(low level capabilities)}: primitive tasks
         \pause
         
         \textit{\footnotesize(move, grasp, look)}
         \pause
         \item[$T$] \textit{(high-level capabilities)}: abstract tasks
         \pause

         \textit{\footnotesize(set the table, prepare a coffee)}
         \pause
         \item[$M_t$] \textit{(Operational models)}: methods
         ("know how")
         \footnotesize

         \textbf{Executable program: } General purpose language, common programming constructs
     \end{itemize}
    \end{column}
    \begin{column}{0.45\textwidth}
        \begin{figure}
            \begin{tikzpicture}
                \node[draw, rounded corners, ultra thick] (t) {$t_{move}$} [sibling distance = 2.5cm]
                  child {node[draw, ultra thick] (m1) {$m_{noop}$} edge from parent [dashed]} 
                  child {node[draw, ultra thick] (m2) {$m_{recursive}$} edge from parent [dashed]
                  child {node[draw, circle, solid, ultra thick] (a1) {move} edge from parent [solid]}
                  child {node[draw, rounded corners, solid, ultra thick] {$t_{move}$} edge from parent [solid]}};
                \node[right = 0em of t] {$\in T$};
                \node[right = 0em of m1] {$\in M_t$};
                \node[right = 0em of m2] {$\in M_t$};
                \node[right = 0em of a1] {$\in A$};

            \end{tikzpicture}
            \caption{Example of hierarchy for the \textit{task} $t_{move}$}

            
        \end{figure}
    \end{column}
\end{columns}
    
\end{frame}
\begin{frame}{Refinement Acting Engine (RAE) : Algorithms to handle acting domains and make choices}
\begin{columns}
    \begin{column}{0.35\textwidth}
    \pause
    \setlength{\leftmargini}{-1pt}
    %\setlength{\parsep}{1pt}
    %\setlength{\parskip}{0pt}
    \begin{itemize}
        \item Performs multiple tasks in parallel
        \pause
        \item RAE decision points:
        \begin{itemize}
            %\setlength{\leftmargini}{-1cm}
            \pause
            \item \textbf{Refinement:}
            
            $task \rightarrow  method \in M_t$.
            \pause
            \item \textbf{Instantiating:}
            
            $parameter \rightarrow value \in D_p$
        \end{itemize}
    \end{itemize}
    \end{column}
    \begin{column}{0.65\textwidth}
        \begin{itemize}
            \setlength{\leftmargini}{-1pt}
            \item \textbf{Main:} 
            \begin{itemize}
                \item Receive:  $\tau$ (task or event);
                add it the \textbf{agenda} ({ongoing tasks})
                \pause
                \item Refine: $tau$ using \textbf{Select.}
                \pause
                \item Progress: $tau \in$ agenda
            \end{itemize}
            \pause
            \item Select: $\tau \rightarrow m_s$
            
            $ m_s \in \{m \in M_t, applicable(m)\}$
            \pause
            \item \textbf{Progress:}
                \begin{itemize}
                \item Refine subtasks $\tau_i$ in $m_s$.    
                \item Monitor execution of $\tau_i$.
                \item \textbf{Retry} the $\tau$ in case of \emph{failure}.
                \end{itemize}
                \pause
            
            \item \textbf{Retry:} call Select to get a new method; progress the new method.
        \end{itemize}
    \end{column}
\end{columns}

    
\end{frame}
\begin{frame}{Using planning to guide the refinement}
    Make a high-level choice based on future choices the agent will have to make and its own capabilities to modify its environment.
    Requirements:
    \begin{itemize}
        \item a Model
        \item \dots
    \end{itemize}
    Problems: 
    \begin{itemize}
        \item 
    \end{itemize}
    Proposed solution:
    \begin{itemize}
        \item 
    \end{itemize}
\end{frame}
\begin{frame}{Improve Select with planning}
    \begin{center}
        
    Role:
    
    $Select(\tau, p_1,\dots,p_n) \rightarrow \{m, p_1, \dots, p_n,\dots,p_m\}$

    \end{center}

    Techniques:
    \begin{itemize}

    \item \textbf{Greedy (Basic RAE functioning):} $\rightarrow m_s \in applicable(m)$
    \pause
    
    \underline{Problem:} Does not take into account future refinements (can lead into dead-locks)
    \pause
    
    \item \textbf{Heuristic:} sort with heuristic function
    \pause
    
    \underline{Propositions: } Learn$\pi$ and Learn$\pi_i$
    \pause
    
    \underline{Problem :} Gives better quality choices but does not ensures dead-lock free choices
    \pause
    \item \textbf<5->{Use look-ahead techniques} : capacity to project the system from the current state to possible future state

    \end{itemize}
\end{frame}

\begin{frame}{Extract a planning domain from acting domain}
    Use a Lisp language:
    \begin{itemize}
        \item Turing complet
        \item primitives d'acting
        \item immutable
        \item functional
        \item pure
    \end{itemize}
\end{frame}