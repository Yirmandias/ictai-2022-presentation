\section{Acting for multi-agents}
\subsection{OMPAS: Extending RAE for fleet management}
\begin{frame}{OMPAS: A new implementation of RAE}
    
            RAE extension:
    \begin{itemize}
        \item Parallel task handled in dedicated threads
        \item Dedicated language with support for
        \begin{itemize}
            \item Concurrency
            \item General programming constructs
            \item Error and interruption
        \end{itemize}
    \end{itemize}  
    
\end{frame}

\begin{frame}{OMPAS main algorithm}
    \begin{columns}
        \begin{column}{0.4\textwidth}
            \begin{itemize}
                \item Each task is evaluated in its own thread.
                \item Shared objects among all threads: state, resources
            \end{itemize}        
        \end{column}
        \begin{column}{0.6\textwidth}
            \tikzstyle{thread} = [draw, rounded corners, rectangle, fill=black!50]
    \begin{tikzpicture}[scale=0.5, every node/.style={scale=0.5}]
  
        %RAE
        \node[draw = black, very thick,
            minimum width = 14em,
            minimum height = 5em,
            rounded corners,
            text = black,
            text centered, text depth = 4 cm,
            align = center,
            ] (R) at (0,0) {\textbf{OMPAS}};
        %Operational Models
        % \node[draw = black, ellipse, very thick,
        %     minimum width = 7em,
        %     minimum height = 2em,
        %     above = 1em of R,
        %     text = black,
        %     text centered, align=center
        % ] (OM) {\textit{Operational Models} %($A_\Delta$)
        % };
  
  
        %state
        \node[draw = black, very thick,
        text centered,
        minimum width = 13em,
        minimum height = 2em,
        rounded corners,
        text = black,
        fill=black!10,
        align = center] (S) at ($(R.south) + (0,0.6)$) {State};
  
        %main
        \node[draw = black, very thick,
        minimum height = 8em,
        minimum width = 13em,
        rounded corners,
        text = black,
        text centered, text depth = 7em,
        fill=black!10,
        align=center] (main) at ($(R.north) + (0,-2)$) {Main};
  
        %t1
        \node[draw = black, thick,
        minimum height = 6em,
        minimum width = 7em,
        rounded corners,
        text = black,
        text centered, text depth = 5em,
        fill=black!20,
        align=center](t1) at ($(main.south) + (-0.8, 1.3)$) {\emph{transport($o_1$,$l_7$)}};
  
  
        %first task
        \node[draw, rounded corners, rectangle, fill=black!30] (st11) at ($(t1.south) + (0,1.3)$) {\emph{drive($t_1$,$l_8$)}};
        \node[draw, rounded corners, rectangle, fill=black!30] (st12) at ($(st11.south) + (0,-.5)$){\emph{pack($o_1$)}};
        
        \node[draw, thick, rounded corners, rectangle,minimum width = 4em, fill=black!20, align= center] (t3) at ($(t1.east) + (0.9, 0.3)$){\textit{monitor}\\
        \textit{-fuel}($t_1$)};
        \node[draw, thick, rounded corners, rectangle,minimum width = 4em, fill=black!20, align= center] (t3) at ($(t1.east) + (0.9, -0.5)$){\dots};
  
  
  
        %platform
        \node[draw = black, very thick,
        text centered,
        minimum width = 5em,
        minimum height = 2em,
        %below = 2em of R,
        rounded corners,
        text = black,
        align = center,] (Pl) at ($(R.west) +(-8em, -2em)$) {Platform};
  
        \node[text centered,
        minimum width = 5em,
        minimum height = 2em,
        below = 1.5em of Pl,
        rounded corners,
        text = black,
        align = center,] (E) {\textbf{\large\textit{Environment}}};
  
        \node[very thick,
            text centered,
            rounded corners,
            text = black] (U) at ($(R.west) +(-8em, +4em)$) {\Large \textbf{User}};
        
        % \node[draw = black, very thick,
        %     text centered,
        %     right = 7em of R,
        %     minimum width = 5em,
        %     minimum height = 4em,
        %     rounded corners,
        %     text = black] (Pl) {Platform};
        
  
        \path[->]
  
            %Link between operational models library
            %(OM) edge (R)
  
            (U.20) edge[bend left] node [midway, above, align=center] {\footnotesize \emph{Tasks}} (main.+140)
            (R.+160) edge [bend left] node [midway, above] {\footnotesize \emph{Reports}} (U.-20)
        
            %Link between RAE and the platform
            (main.-160) edge[bend left] node [pos = 0.6, above = 0.2em] {\footnotesize \emph{Commands}} (Pl.east)
            
            %between platform and state
            (Pl.-20) edge [bend right] node [pos = 0.7, right, above, align = center] {\footnotesize \emph{Updates}} (S.west)
            
            %between platform and main
            (Pl.20) edge [bend left] node [pos = 0.3, right, above= 0.2em, align = center] {\footnotesize \emph{Events}} (main.-170)
  
            (Pl.-110) edge (E.110)
            (E.70) edge (Pl.-70)
            ;
        
        \end{tikzpicture}
        \end{column}
    \end{columns}
\end{frame}

\subsection{SOMPSA: Dedicated acting language}

\begin{frame}{SOMPAS: A Lisp variant with acting primitives}
    
    \begin{columns}
        \begin{column}{0.5\textwidth}
            Why Scheme?
            \begin{itemize}
                \item Few primitives
                \item Functional
                \item Generic programming constructs
                \item Identified and simple semantic $\rightarrow$ suitable to automated analysis for planning
            \end{itemize}
        \end{column}
        \begin{column}{0.5\textwidth}
            Features :
            \begin{itemize}
                \item Concurrency and interruption
                \item Acting primitives
                \item Resource acquisition mechanism for synchronization and exclusion.
            \end{itemize}

        \end{column}
    \end{columns}
\end{frame}

\begin{frame}[fragile]{SOMPAS: Concurrency and interruption}
    handle: new LValue
    Functions:
    \begin{itemize}
        \item \textit{(async e)}: creates a new thread to evaluate e, returns a handle
        \item \textit{(await h)}: awaits the result of the concurrent evaluation.
        \item \textit{(interrupt h)}: interrupts the concurrent evaluation.
    \end{itemize}
    
\begin{columns}
    \begin{column}{0.4\textwidth}
        \lstset{basicstyle=\footnotesize, columns=fullflexible}
\begin{lstlisting}[language=lisp]
(begin
 (define h
  (async 
    (begin
     (uninterruptible (begin
       (exec pick ?r ?o)
       (exec move ?r ?l)
       (exec drop ?r ?o)))
     (exec inspect ?r ?o))))
 (race 
   (await h)
   (begin
    (sleep 10)
    (interrupt h))))
\end{lstlisting}

    \end{column}
    \begin{column}{0.6\textwidth}
        \newcommand{\Cross}{$\mathbin{\tikz [x=1.4em,y=1.4em,line width=.2em, red] \draw (0,0) -- (1,1) (0,1) -- (1,0);}$}
    \begin{tikzpicture}
        \node[draw,
        rounded corners,
        ultra thick,
        minimum height = 3em,
            minimum width = 7em,] (pick) at (0,0) {pick r1 o3};
    
        \node[draw, rounded corners,
        ultra thick,
            minimum height = 3em,
            minimum width = 7em,
        ] (move) at ($1.5*(pick.south)-0.5*(pick.north)$) {move r1 l6};
        %\action{move, move r1 l6, 10em, pick}
    
        \node[draw, rounded corners,
        ultra thick,
            minimum height = 3em,
            minimum width = 7em,
        ] (drop) at ($1.5*(move.south) - 0.5*(move.north)$) {drop r1 o3};
        \node[draw, rounded corners,
        ultra thick,
            minimum height = 3em,
            minimum width = 7em,
        ] (inspect) at ($1.5*(drop.south) - 0.5*(drop.north)$) {inspect r1 o3};
        \node at (inspect) {\Cross};
    
            \node[] (origin) at ($(pick.north west) + (-1em,+0.5em)$) {~~};
        
            \node[] (zero) at ($(origin.west) + (-1em,0)$) {0};
            \draw[-, ultra thick] (origin.west) -- (origin.east);
            \node[] (ten) at ($(zero) + (0,-6em)$) {10};
            \draw[-, ultra thick] ($(origin.west) + (0,-6em)$) -- ($(origin.east) + (0,-6em)$);
    
    
            \node[] (interruption) at ($(ten.east) +(11em,0em)$){\color{red} \textit{Int.}};%{\color{red} \textit{interruption}};
            \draw[-, ultra thick, dashed, color= red] ($(ten.east)$) -- ( interruption);
            \node[] (end) at ($1.5*(inspect.south west) - 0.5*(inspect.south west) + (-1em,-1em)$) {};
            \node[] (time) at ($1.5*(end.south east) - 0.5*(end.south east) + (2em,0)$)  {time (seconds)};
    
    
        \path[->] (origin.center) edge[ultra thick] (end);
    \end{tikzpicture}
    \end{column}
\end{columns}    
\end{frame}

\begin{frame}{SOMPAS: Acting primitives}
    \begin{itemize}
    \pause
        \item \textit{(exec a $p_1...p_n$)} executes and monitors a task or command.
    \pause
        \begin{itemize}
            \item command : resorts to the platform
            \item task: selects a method and execute its operational model.
        \end{itemize}
    \pause
        \item \textit{(read-state sf $p_1...p_n$)} returns the value of a state-variable
        \item \textit{(arbitrary set $\lambda$)} returns an arbitrary element from a $set$. $\lambda$ can be used to select among the $set$.
    \end{itemize}
\end{frame}

\begin{frame}[t]{SOMPAS: Resources}
    \begin{columns}
        \begin{column}{0.5\textwidth}
            Definition of a resource:
            \begin{itemize}
                \item Object with initial capacity $C_{init}$
                \item Acquire $r$ at $t$ with amount $c$: $c \leq C_t$ 
            \end{itemize}
            Kinds of resources : 
            \begin{itemize}
                \item \emph{unary}: $C_{init} = 1$, $c = 1$;
                \item \emph{divisible}: $C_{init} \in  \mathbb{R}_+^*$, $c \in ]0, C_t]$
            \end{itemize}
        \end{column}
        \begin{column}{0.5\textwidth}
            Functions
            \begin{itemize}
                \item \textit{(new-resource r C)}: declare a new resource with a label and a capacity.
                \item \emph{(acquire r c)} request the acquisition of $r$ with amount $c$. Returns a resource-handle
                \item \emph{(release h)} release the resource of the resource-handle
            \end{itemize}
        \end{column}
    \end{columns}
\end{frame}

% \begin{frame}{More advanced features and programming constructs}
%     More complex behavior definition
%     \begin{itemize}
%         \item \textit{(wait-for dyn)} waits until the expression \textit{dyn} becomes true 
%         \item \textit{(run-monitoring e dyn)} evaluates \textit{e} while \textit{dyn} is true, interrupts \textit{e} otherwise.
%         \item ...
%     \end{itemize}
% \end{frame}

\begin{frame}[t,fragile]{Defining an acting domain with SOMPAS}
    \setlength{\leftmargini}{0pt}
    \begin{columns}
        \begin{column}{0.5\textwidth}
            \begin{itemize}
                \footnotesize
                \item Task (label, typed parameters):
                \tiny
                \begin{lstlisting}
(def-task t_process_package
    (:params (?p package)))
                \end{lstlisting}
                \footnotesize
                \pause
                \item Method (label, typed parameters, pre-conditions, body):
                \tiny
                \begin{lstlisting}
(def-method m_process_to_do_r
  (:task t_process_package)
  (:params (?p package))
  (:pre-conditions
    (!= (package.processes_list ?p)
        nil)))
  (:body
    (do
     (define ?m ...)
     (t_process_on_machine ?p ?m)
     (t_process_package ?p)))   
                \end{lstlisting}
            \end{itemize}
        \end{column}
    \pause
        \begin{column}{0.5\textwidth}
            \begin{itemize}
    \pause
            \footnotesize
            \item State-function (label, typed parameters \& result):
            \tiny
            \begin{lstlisting}
(def-state-function at
    (:params (?r robot))
    (:result location))
            \end{lstlisting}
    \pause
            \footnotesize
            \item Command (label, typed parameters):
            \tiny
            \begin{lstlisting}
(def-command pick (:params (?r robot)))
                \end{lstlisting}
            \end{itemize}

    \pause
            ~
        \normalsize
        Others:
        
        \textit{lambda, type, constant}
        \end{column}

    \end{columns}
\end{frame}