\section{RAE for multi-agent systems}
\subsection{Operational Model Planning and Acting System}

\begin{frame}{The Operational Model Planning and Acting System (OMPAS), an extension to RAE}
    \begin{columns}[t]
        \begin{column}{0.5\textwidth}
    New task in separated thread
    \begin{figure}
        \tikzstyle{thread} = [draw, rounded corners, rectangle, fill=black!50]
    \begin{tikzpicture}[scale=0.5, every node/.style={scale=0.5}]
  
        %RAE
        \node[draw = black, very thick,
            minimum width = 14em,
            minimum height = 5em,
            rounded corners,
            text = black,
            text centered, text depth = 4 cm,
            align = center,
            ] (R) at (0,0) {\textbf{OMPAS}};
        %Operational Models
        % \node[draw = black, ellipse, very thick,
        %     minimum width = 7em,
        %     minimum height = 2em,
        %     above = 1em of R,
        %     text = black,
        %     text centered, align=center
        % ] (OM) {\textit{Operational Models} %($A_\Delta$)
        % };
  
  
        %state
        \node[draw = black, very thick,
        text centered,
        minimum width = 13em,
        minimum height = 2em,
        rounded corners,
        text = black,
        fill=black!10,
        align = center] (S) at ($(R.south) + (0,0.6)$) {Estimated State};
  
        %main
        \node[draw = black, very thick,
        minimum height = 8em,
        minimum width = 13em,
        rounded corners,
        text = black,
        text centered, text depth = 7em,
        fill=black!10,
        align=center] (main) at ($(R.north) + (0,-2)$) {Main};
  
        %t1
        \node[draw = black, thick,
        minimum height = 6em,
        minimum width = 7em,
        rounded corners,
        text = black,
        text centered, text depth = 5em,
        fill=black!20,
        align=center](t1) at ($(main.south) + (-0.8, 1.3)$) {\emph{transport($o_1$,$l_7$)}};
  
  
        %first task
        \node[draw, rounded corners, rectangle, fill=black!30] (st11) at ($(t1.south) + (0,1.3)$) {\emph{drive($t_1$,$l_8$)}};
        \node[draw, rounded corners, rectangle, fill=black!30] (st12) at ($(st11.south) + (0,-.5)$){\emph{pack($o_1$)}};
        
        \node[draw, thick, rounded corners, rectangle,minimum width = 4em, fill=black!20, align= center] (t3) at ($(t1.east) + (0.9, 0.3)$){\textit{monitor}\\
        \textit{-fuel}($t_1$)};
        \node[draw, thick, rounded corners, rectangle,minimum width = 4em, fill=black!20, align= center] (t3) at ($(t1.east) + (0.9, -0.5)$){\dots};
  
  
  
        %platform
        \node[draw = black, very thick,
        text centered,
        minimum width = 5em,
        minimum height = 2em,
        %below = 2em of R,
        rounded corners,
        text = black,
        align = center,] (Pl) at ($(R.west) +(-8em, -2em)$) {Platform};
  
        \node[text centered,
        minimum width = 5em,
        minimum height = 2em,
        below = 1.5em of Pl,
        rounded corners,
        text = black,
        align = center,] (E) {\textbf{\large\textit{Environment}}};
  
        \node[very thick,
            text centered,
            rounded corners,
            text = black] (U) at ($(R.west) +(-8em, +4em)$) {\Large \textbf{User}};
        
        % \node[draw = black, very thick,
        %     text centered,
        %     right = 7em of R,
        %     minimum width = 5em,
        %     minimum height = 4em,
        %     rounded corners,
        %     text = black] (Pl) {Platform};
        
  
        \path[->]
  
            %Link between operational models library
            %(OM) edge (R)
  
            (U.20) edge[bend left] node [midway, above, align=center] {\footnotesize \emph{Tasks}} (main.+140)
            (R.+160) edge [bend left] node [midway, above] {\footnotesize \emph{Reports}} (U.-20)
        
            %Link between RAE and the platform
            (main.-160) edge[bend left] node [pos = 0.6, above = 0.2em] {\footnotesize \emph{Commands}} (Pl.east)
            
            %between platform and state
            (Pl.-20) edge [bend right] node [pos = 0.7, right, above, align = center] {\footnotesize \emph{Updates}} (S.west)
            
            %between platform and main
            (Pl.20) edge [bend left] node [pos = 0.3, right, above= 0.2em, align = center] {\footnotesize \emph{Events}} (main.-170)
  
            (Pl.-110) edge (E.110)
            (E.70) edge (Pl.-70)
            ;
        
        \end{tikzpicture}
        \caption{Overview of OMPAS architecture}
    \end{figure}
            
        \end{column}
        \pause
        \begin{column}{0.5\textwidth}
        Operational language with
        \begin{itemize}
            \item Acting features
            \item Generic programming constructs
            \item First-hand concurrency support (interruption, resource)
            \item Simple semantic $\rightarrow$ guide the acting choices with automated analysis of skills
        \end{itemize}
        \end{column}
    \end{columns}
\end{frame}

\subsection{Acting language}

\begin{frame}[fragile]{SOMPAS: A Lisp variant based on Scheme}
    \centering
    Scheme : Recursive evaluation of expressions \verb|(f e1 ... en)|
    
    ~~

\pause
    \begin{columns}[t]
        \begin{column}{0.65\textwidth}
            Definitions:
            \begin{itemize}
                \item Expression = \{Atom, Expression List\}
                \item Atom = \{Symbol, Boolean, Number, Procedure\}
            \end{itemize}
\pause
        \end{column}
        \begin{column}{0.35\textwidth}
            Scheme advantages:
            \begin{itemize}
                \item Few primitives
                \item Functional
                \item Simple to extend
            \end{itemize}
        \end{column}
     \end{columns}
\end{frame}

\begin{frame}[fragile]{Acting primitives}
    \begin{columns}
        \begin{column}{0.5\textwidth}
            Functions:
            \begin{itemize}
                \pause
                    \item \textbf{\textit{(exec a $p_1...p_n$)}} executes and monitors a task or command.
                    % \begin{itemize}
                    %     \item command : resorts to the platform
                    %     \item task: selects a method and execute its operational model.
                    % \end{itemize}
                \pause
                    \item \textbf{\textit{(read-state sf $p_1...p_n$)}} returns the value of a state-variable
                % \pause
                %     \item \textit{(arbitrary set $\lambda$)} returns an arbitrary element from a $set$%. $\lambda$ can be used to select among the $set$.
                \end{itemize}
        \end{column}
    \pause
        \begin{column}{0.5\textwidth}
               
        Example:
    \small
        \lstset{columns=fullflexible}
        \begin{lstlisting}[language = lisp]
    (begin
        (define loc-p (read-state loc ?p))
        (define loc-r (read-state loc ?r))
        (if (!= loc-p loc-r)
            (exec move ?r loc-p))
        (exec pick ?r ?p) 
        (exec move ?r ?l)
        (exec place ?r ?p))             
        \end{lstlisting}
        \end{column}
    \end{columns}
    \end{frame}


\begin{frame}[fragile]{Scheme extension for concurrency and interruption}

    Definition of a new Atom: the \emph{handle}
    
    ~~
\pause
    \begin{columns}
        \begin{column}{0.6\textwidth}
            Functions:
            \begin{itemize}
                \item \textbf{\textit{(async e)}}: creates a new thread to evaluate e, returns a handle
                \item \textbf{\textit{(await h)}}: awaits the result of the concurrent evaluation.
                \item \textbf{\textit{(interrupt h)}}: interrupts the concurrent evaluation.
                \item \textbf{\textit{(uninterruptible e)}}: prevent interruption signals.
            \end{itemize}
        \end{column}
        \pause
        \begin{column}{0.4\textwidth}
        Example: 
\small
\lstset{columns=fullflexible}
            \begin{lstlisting}[language = lisp]
(begin
    (define h1 (async (+ 1 2)))
    (define h2 (async (* 3 3)))
    (+ (await h1) (await h2)))
-> 12 
            \end{lstlisting}
        \end{column}
    \end{columns}
\end{frame}

\begin{frame}[fragile]{Example of program with interruptions.}
    
\begin{columns}
    \begin{column}{0.4\textwidth}
        \lstset{basicstyle=\footnotesize, columns=fullflexible}
\begin{lstlisting}[language=lisp]
(begin
 (define h
  (async 
    (begin
     (uninterruptible (begin
       (exec pick ?r ?o)
       (exec move ?r ?l)
       (exec drop ?r ?o)))
     (exec inspect ?r ?o))))
 (race 
   (await h)
   (begin
    (sleep 10)
    (interrupt h))))
\end{lstlisting}

    \end{column}
    \begin{column}{0.6\textwidth}
        \newcommand{\Cross}{$\mathbin{\tikz [x=1.4em,y=1.4em,line width=.2em, red] \draw (0,0) -- (1,1) (0,1) -- (1,0);}$}
    \begin{tikzpicture}
        \node[draw,
        rounded corners,
        ultra thick,
        minimum height = 3em,
            minimum width = 7em,] (pick) at (0,0) {pick r1 o3};
    
        \node[draw, rounded corners,
        ultra thick,
            minimum height = 3em,
            minimum width = 7em,
        ] (move) at ($1.5*(pick.south)-0.5*(pick.north)$) {move r1 l6};
        %\action{move, move r1 l6, 10em, pick}
    
        \node[draw, rounded corners,
        ultra thick,
            minimum height = 3em,
            minimum width = 7em,
        ] (drop) at ($1.5*(move.south) - 0.5*(move.north)$) {drop r1 o3};
        \node[draw, rounded corners,
        ultra thick,
            minimum height = 3em,
            minimum width = 7em,
        ] (inspect) at ($1.5*(drop.south) - 0.5*(drop.north)$) {inspect r1 o3};
        \node at (inspect) {\Cross};
    
            \node[] (origin) at ($(pick.north west) + (-1em,+0.5em)$) {~~};
        
            \node[] (zero) at ($(origin.west) + (-1em,0)$) {0};
            \draw[-, ultra thick] (origin.west) -- (origin.east);
            \node[] (ten) at ($(zero) + (0,-6em)$) {10};
            \draw[-, ultra thick] ($(origin.west) + (0,-6em)$) -- ($(origin.east) + (0,-6em)$);
    
    
            \node[] (interruption) at ($(ten.east) +(11em,0em)$){\color{red} \textit{Int.}};%{\color{red} \textit{interruption}};
            \draw[-, ultra thick, dashed, color= red] ($(ten.east)$) -- ( interruption);
            \node[] (end) at ($1.5*(inspect.south west) - 0.5*(inspect.south west) + (-1em,-1em)$) {};
            \node[] (time) at ($1.5*(end.south east) - 0.5*(end.south east) + (2em,0)$)  {time (seconds)};
    
    
        \path[->] (origin.center) edge[ultra thick] (end);
    \end{tikzpicture}
    \end{column}
\end{columns}    
\end{frame}


\begin{frame}[fragile]{Resource model for safe concurrent execution}
    Definition of a resource:
    \begin{columns}
        \begin{column}{0.5\textwidth}
            \begin{itemize}
                \item Object \textbf{r} with initial capacity $C(0) = C_{init} $
                \item Acquire($r$,$c$) at time $t$ $\implies$ $c \leq C(t)$ 
            \end{itemize}
        \end{column}
        \pause
        \begin{column}{0.5\textwidth}
            \begin{itemize}
                \item \textbf{\textit{unary}:} $C_{init} = 1, c = 1$;
                \item \textbf{\textit{divisible}}: $C_{init} \in  \mathbb{R}_+^*, c \in ]0, C(t)]$
            \end{itemize}
        \end{column}
    \end{columns}
    
    ~~

\pause
    \begin{columns}
        \begin{column}{0.6\textwidth}
            Functions:
            \begin{itemize}
                \item \textbf{\textit{(new-resource r C)}} : create a new resource $r$ with capacity $C$
                \item \textbf{\textit{(acquire r c)}} : acquisition request for $r$ with amount $c$, returns a resource handle
                \item \textbf{\textit{(release h)}} : release the resource using the resource handle
            \end{itemize}
        \end{column}
        \pause
        \begin{column}{0.4\textwidth}
        Example: 
            \small
            \lstset{columns=fullflexible}
            \begin{lstlisting}[language = lisp]
(begin
 (new-resource r1)
 ...
 (define hr (acquire r1))
 (exec move r1 x y)
 (release hr)
)
            \end{lstlisting}
        \end{column}
    \end{columns}
\end{frame}

% \begin{frame}{More advanced features and programming constructs}
%     More complex behavior definition
%     \begin{itemize}
%         \item \textit{(wait-for dyn)} waits until the expression \textit{dyn} becomes true 
%         \item \textit{(run-monitoring e dyn)} evaluates \textit{e} while \textit{dyn} is true, interrupts \textit{e} otherwise.
%         \item ...
%     \end{itemize}
% \end{frame}

% \begin{frame}[t,fragile]{Defining an acting domain with SOMPAS}
%     \setlength{\leftmargini}{0pt}
%     \begin{columns}
%         \begin{column}{0.5\textwidth}
%             \begin{itemize}
%                 \footnotesize
%                 \item Task (label, typed parameters):
%                 \tiny
%                 \begin{lstlisting}
% (def-task t_process_package
%     (:params (?p package)))
%                 \end{lstlisting}
%                 \footnotesize
%                 \pause
%                 \item Method (label, typed parameters, pre-conditions, body):
%                 \tiny
%                 \begin{lstlisting}
% (def-method m_process_to_do_r
%   (:task t_process_package)
%   (:params (?p package))
%   (:pre-conditions
%     (!= (package.processes_list ?p)
%         nil)))
%   (:body
%     (do
%      (define ?m ...)
%      (t_process_on_machine ?p ?m)
%      (t_process_package ?p)))   
%                 \end{lstlisting}
%             \end{itemize}
%         \end{column}
%     \pause
%         \begin{column}{0.5\textwidth}
%             \begin{itemize}
%     \pause
%             \footnotesize
%             \item State-function (label, typed parameters \& result):
%             \tiny
%             \begin{lstlisting}
% (def-state-function at
%     (:params (?r robot))
%     (:result location))
%             \end{lstlisting}
%     \pause
%             \footnotesize
%             \item Command (label, typed parameters):
%             \tiny
%             \begin{lstlisting}
% (def-command pick (:params (?r robot)))
%                 \end{lstlisting}
%             \end{itemize}

%     \pause
%             ~
%         \normalsize
%         Others:
        
%         \textit{lambda, type, constant}
%         \end{column}

%     \end{columns}
% \end{frame}

